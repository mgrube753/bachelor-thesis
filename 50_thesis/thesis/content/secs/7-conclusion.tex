\section{Conclusion and Future Work} \label{sec:conclusion-and-future-work}

% \newtodo{Brief overview of everything i have done.}
% \newtodo{Provide an outlook on future work planned to do or at least thoughts besides implementation.}

\subsection{Key Findings}

% with strengths and weaknesses

\vp

\subsection{Practical Implications for Education}

% The findings of this thesis offer several practical implications for the field of education. The demonstrated ability of \acp{llm} to generate varied question types aligned with Bloom's Taxonomy can significantly reduce the workload for educators in creating assessment materials. This allows lecturers to focus more on direct student interaction and personalized feedback. Furthermore, the insights into prompt engineering for targeted question generation can empower educators to leverage \acp{llm} for creating customized learning exercises that cater to specific cognitive levels and learning objectives, potentially enhancing student engagement and understanding. The potential for generating questions that accurately reflect provided instructional content opens avenues for scalable and adaptive learning systems.
% the features in the created prompts are not only useful for the \ac{llms}, but also for the educators, as they can use them to create their own prompts. 

\vp

\subsection{Future Research Directions}

% While this study provides valuable insights, several avenues for future research remain. Firstly, refining the evaluation metrics for \ac{llm}-generated questions, particularly for higher-order cognitive skills as defined by Bloom's Taxonomy, is crucial. Developing more nuanced automated evaluation techniques that capture pedagogical suitability beyond content accuracy would be beneficial. Secondly, extending the research to include different types of educational content, such as mathematical problems or visual materials, could broaden the applicability of these findings. 
% multimodal models are uprising, so this could be a good idea to explore
% Investigating the integration of these question-generation capabilities into existing \ac{lmss} could streamline their practical adoption. Finally, exploring the long-term impact of using \ac{llm}-generated questions on student learning outcomes and engagement warrants further study.


% \object{Refinement of Evaluation Criteria}

% \object{Integration with Learning Management Systems}
