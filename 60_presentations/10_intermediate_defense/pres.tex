\documentclass[usenames, dvipsnames, aspectratio=32]{beamer}
\usepackage[utf8]{inputenc}
\usepackage[T1]{fontenc}
\usepackage[german]{babel}
\usepackage[defaultfam, regular]{montserrat}
\usepackage{csquotes}
\usepackage[normalem]{ulem}
\usepackage{latexsym,multicol,booktabs,miama}
\usepackage{setspace}
\usepackage{xcolor}
\usepackage{tikz}
\usepackage{amssymb,amsfonts,amsmath,amsthm,mathrsfs,mathptmx} 
\usepackage{graphicx,pstricks,listings,stackengine}
\usepackage[backend=biber,bibencoding=utf8,urldateusetime=true,style=alphabetic,sorting=nyt,natbib=true]{biblatex}
\usepackage{style}
\usepackage{hyperref}
    \hypersetup{colorlinks=true, citecolor=purple, filecolor=black, linkcolor=main, urlcolor=blue}
\setbeamertemplate{bibliography item}{\insertbiblabel}
% \addbibresource{ref.bib}

\author[Malte Grube]{\vspace*{-.8em}Malte Grube -- B.\,Sc. Informatik (2021)\inst{1}}
\title[]{LLMs in der Fragengenerierung}
\subtitle{--\;Zwischenverteidigung\;--}
\institute[Universität Rostock]{
    \inst{1}
    \footnotesize{Betreuer: \textit{Prof.\,Dr. rer.nat.habil. \uline{Clemens H. Cap}}} \vspace*{.25em} \\
    \textit{Jun.-Prof.in Dr.in \uline{Charlott Rubach}} \vspace*{-.25em}
}
\date{15. Oktober 2024}

\def\cmd#1{\texttt{\color[RGB]{0, 0, 139}\footnotesize $\backslash$#1}}
\def\env#1{\texttt{\color[RGB]{0, 0, 139}\footnotesize #1}}

\lstdefinestyle{texstyle}{
    language=[LaTeX]TeX,
    basicstyle=\ttfamily\footnotesize,
    keywordstyle={\bfseries\color[RGB]{0, 0, 180}},
    keywordstyle=[2]{\slshape\color[RGB]{225, 140, 30}},
    morekeywords=[2]{,itemize, enumerate, equation, table, tabular},
    stringstyle=\color[RGB]{50, 50, 50},
    numbers=left,
    numberstyle=\footnotesize\color{gray},
    rulesepcolor=\color{red!20!green!20!blue!20},
    frame=shadowbox
}

% \AtBeginSection[]{
%   \begin{frame}
%     \tableofcontents[sectionstyle=show/shaded,subsectionstyle=show/shaded/hide,subsubsectionstyle=show/shaded/hide]
%   \end{frame}
% }

\begin{document}
{
\setbeamertemplate{headline}{}
\begin{frame}
    \vspace*{1em}
    \begin{figure}[htpb]
    \centering
    \includegraphics[width=0.55\linewidth]{logo.jpg}
    \end{figure}
    \vspace*{-1.5em}
    \titlepage
\end{frame}
}

\begin{frame}
    \tableofcontents[sectionstyle=show,subsectionstyle=show/shaded/hide,subsubsectionstyle=show/shaded/hide]
\end{frame}

\section{Einführung \& Problem}

\begin{frame}{Motivation \& Problemstellung}
    \begin{itemize}
        \item Test 
    \end{itemize}
\end{frame}

\begin{frame}{Forschungsfragen}
    \begin{itemize}
        \item Test 
    \end{itemize}
\end{frame}

\begin{frame}{Bloom's Taxonomy in der Bildung}
    \begin{itemize}
        \item Test 
    \end{itemize}
\end{frame}

\begin{frame}{LLMs in der Fragengenerierung}
    \begin{itemize}
        \item sota of my papers in thesis.tex 
    \end{itemize}
\end{frame}

\section{Exp.\ 1 Methodik}

\begin{frame}{Experimentelles Design}
    \begin{itemize}
        \item Aufbau und Ziele von Exp1
        \item Source Materials
    \end{itemize}
\end{frame}

\begin{frame}{Datenerhebung \& Durchführung}
    \begin{itemize}
        \item Python, embedding modell, apis ausser deepseek, automatisierte anfragen mit diversem prompting, diversen source materials, ...
    \end{itemize}
\end{frame}

\begin{frame}{Evaluations-Methodik}
    \begin{itemize}
        \item quantitative mit metriken (semantic sim, adherence via llm)
        \item qualitative mit cap als experte ggü. llm as an evaluator (criteria) 
    \end{itemize}
\end{frame}

\section{Ergebnisse \& Diskussion}

\begin{frame}{Hypothesen}
    \begin{itemize}
        \item Null- und Alternativhypothese! %#Todo
    \end{itemize}
\end{frame}

\begin{frame}{Quantitative Analyse}
    \begin{itemize}
        \item Hier müssen viele Tabellen und Grafiken rein!
    \end{itemize}
\end{frame}

\begin{frame}{Qualitative Analyse}
    \begin{itemize}
        \item Hier müssen viele Tabellen und Grafiken rein!
    \end{itemize}
\end{frame}

\begin{frame}{Interpretation\,/\,Limitations}
    \begin{itemize}
        \item Hier müssen viele Tabellen und Grafiken rein!
    \end{itemize}
\end{frame}

\begin{frame}{Hypothese vs. Forschungsfrage}
    \begin{itemize}
        \item Test
    \end{itemize}
\end{frame}

\section{Fazit \& Ausblick}

\begin{frame}{Hauptergebnisse}
    \begin{itemize}
        \item Test
    \end{itemize}
\end{frame}

\begin{frame}{Lessons Learned}
    \begin{itemize}
        \item Test
    \end{itemize}
\end{frame}

\begin{frame}{Ausblick auf Exp. 2}
    \begin{itemize}
        \item Test
    \end{itemize}
\end{frame}

\begin{frame}{Timeline bis zur Verteidigung}
    \begin{itemize}
        \item Test
    \end{itemize}
\end{frame}

\end{document}
